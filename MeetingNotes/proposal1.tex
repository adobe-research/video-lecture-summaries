\documentclass{report}
\begin{document}
\noindent\textbf{Making effective navigation for online education videos}\\
\textbf{Making online videos interactive}\\

\textbf{Input style:} classroom lecture, powerpoint lectures, Khan Academy style
\begin{itemize}
\item Different styles may require different edits/techinques for naviagtion. Personally, I think classroom lecture $\geq$ Khan Academy style $\geq$ powerpoint slide lectures in order of difficulty to navigate. 
\item Several works(\cite{Cross2013}, \cite{Cross2014}) suggest that different styles are preferred for different purposes. Also interesting is the idea of \textbf{converting from one style of video to other}.
\end{itemize}

\textbf{Input content:} general vs. select subject 
\begin{itemize}
\item Choosing a specific subject may be more important for adding interactivity, participatory material
\end{itemize}

\textbf{Random Ideas}
\begin{itemize}
\item Navigation of visual content
\begin{itemize}
\item Segmenting videos into meaningful \textit{chapters}
\item Creating static \textit{key frames} out of continuous lectures
\item Linking related parts of videos
\item Creating various length trailers (e.g. 1min, 3min, 10min versions), or summary slides (e.g. 1p, 10p summary notes etc.)
\item We can think about how much to automate/or develop tools that make it easier for users to add any these functions
\end{itemize}
\item Navigation of audio content
\begin{itemize}
\item optimizing audio fastforwarding silent periods, unnecessary content etc. 
\item making searching with audio content easier
\item for content creation and editing: syncing with visual content
\item what would be an equivalent of \textit{key frames for audio}?
\end{itemize}
\item Introducing student interaction
\begin{itemize}
\item I liked Fredo's idea about collecting student output and clustering/visualizing them for the instructor. I think this could provide insight into further steps of providing feedback etc.
\item A related idea is a system to share responses betweens students. Both require some way to cluster and visualize large and varied data from students. 
\item Where and it what form to introduce interaction: hiding parts of contents, letting students finish parts of contents etc.
\end{itemize}
\end{itemize}

\bibliographystyle{plain}
\nocite{*}
\bibliography{education}

\end{document}