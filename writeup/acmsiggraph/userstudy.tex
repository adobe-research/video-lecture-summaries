\section{Evaluation}
To evaluate the utility of our \textit{ScribeVideo} interface, we performed a comparative user study comparing three interfaces to watch lecture videos: a standard YouTube player with an interactive transcript, the NoteVideo interface, and our ScribeVideo interface.  

\subsection{Tasks}
Learners performed three types of tasks for lecture videos: summarization, information search, and comprehension. These tasks represent realistic learning scenarios using video lectures, and match common evaluation tasks applied in the literature on video navigation interfaces \cite{}.
\begin{itemize}
\item \textbf{Summarization} task requires the learners to write down an overview of the main contents in the video in a short period of time. We gave learners only three minutes to view and summarize videos that were about eight minutes long. We purposely did not give enough time to watch the entire video so as to motivate the learners to quickly browse through its content. This is a useful task, for example, when the learner wants to get a quick overview of the lecture without watching the entire video, or when the learner wants to quickly review previously watched material. 
\item \textbf{Search} tasks involve finding a specific piece of information in a video. The questions included both looking for a visual cue (e.g. \textit{"Find the point in the lecture where the height of the graph of $f(x)$ is denoted with the variable $h$."}) and textual cues (e.g. \textit{"Find the point in the lecture where the double angle formula for $cos(2x)$ is stated."}). These tasks emulated situations when a learner remembers something from the lecture and wants to find where it appeared in a video, or situations when the learner only wants to access specific information from the video and skip the rest.  
\item \textbf{Comprehension} tasks involve both finding or identifying relevant information and understanding the content. To gauge knowledge gain from watching videos, we gave learners a 5-question pre-test (before watching) and a 5-question post-test (while watching). The two sets of questions tested the same type of knowledge but were asked in slightly different ways (i.e. one question
would be asked in a \textit{true-or-false} form, and the other question would be in a \textit{fill-in-the-blank} form). The difference in the number of questions answered correctly int he pre- and post- tests was recorded, as well as the task completion time for the post-test.. 
\end{itemize}

\subsection{Study Protocol}
\subsection{Results}
