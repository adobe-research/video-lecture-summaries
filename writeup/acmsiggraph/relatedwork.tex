\section{Related Work}

\textbf{Video Summarization}
Many work on summarizing videos in order to facilitate navigation. \cite{barnes2010video} Video Tapestry, \cite{jackson2013panopticon} Panopticon, \cite{uchihashi1999video}, \cite{hwang2006cinema}\cite{boreczky2000interactive} Comic book style layout of keyframes. Content specific:\cite{ekin2003automatic} soccer video summarization, \cite{} news footage summarization.
 
These algorithms have in common key-frame extraction, and layout. They use features such as color histograms to detect keyframes. Educational video content have different characteristics.
Most relevant to our work is \cite{choudary2007summarization}, which summarizes blackboard-style lectures by creating a panoramic frame of the board.

\textbf{Video Navigation Interface}
\cite{kim2014data} uses interaction data collected from MOOC platforms to introduce a set of techniques that augment existing video interface widgets. \cite{pavel2014video} provides a tool to create \texit{video digests}, structured summaries of informational videos organized into chapters and sections. 
Most closely related to our work is the NoteVideo interface by \cite{monserrat2013notevideo}, which presents a summarized image of the video composed of clickable visual links to specific places in the lecture.

\textbf{Improving Transcripts / Subtitles}
\cite{hu2015speaker} optimizing subtitle placement by speaker recognition. \cite{kurlander1996comic}, \cite{chun2006automated} subtitle placement for single frame in comic style cinema.

\end{itemize}


