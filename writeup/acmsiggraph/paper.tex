%%% template.tex
%%%
%%% This LaTeX source document can be used as the basis for your technical
%%% paper or abstract.

%%% The parameter to the ``documentclass'' command is very important.
%%% - use ``review'' for content submitted for review.
%%% - use ``preprint'' for accepted content you are making available.
%%% - use ``tog'' for technical papers accepted to the TOG journal and
%%%   for presentation at the SIGGRAPH or SIGGRAPH Asia conference.
%%% - use ``conference'' for final content accepted to a sponsored event
%%%   (hint: If you don't know, you should use ``conference.'')


\documentclass[tog]{acmsiggraph}
\usepackage[]{algorithm2e}
\usepackage{amsmath}
\newcommand\todo[1]{\textcolor{red}{#1}}

\DeclareMathOperator*{\argmax}{argmax}
%%% Make the ``BibTeX'' word pretty...

\def\BibTeX{{\rm B\kern-.05em{\sc i\kern-.025em b}\kern-.08em
    T\kern-.1667em\lower.7ex\hbox{E}\kern-.125emX}}

%%% Used by the ``review'' variation; the online ID will be printed on 
%%% every page of the content.

\TOGonlineid{45678}

%%% Used by the ``preprint'' variation.

\TOGvolume{0}
\TOGnumber{0}

\title{ScribeVideo: Static, Linear Re-Presentation of  Blackboard-Style Lecture Videos}

\author{}
\pdfauthor{}

\keywords{Video summarization, Video navigation, Lecture videos, Blackboard-style lectures}

\begin{document}

%%% This is the ``teaser'' command, which puts an figure, centered, below 
%%% the title and author information, and above the body of the content.

 \teaser{
   \includegraphics[width=0.9\textwidth]{images/teaser}
   \caption{ScribeVideo re-presents the contents of blackboard-style lectures in a printable tutorial style, interleaving static figures and text.}
 }

\maketitle

\begin{abstract}


\end{abstract}


\keywordlist

%% Required for all content. 

\copyrightspace
\section{Introduction}
With the introduction of online education platforms such as Khan Academy, Coursera, Udacity, edX, and online courses offered by universities, video watching has become an integral part of the students' learning experience. For example, in 2014, the number of universities offering MOOCs doubled to 400 universities offering over 2,400 courses to 16-18 million enrolled students \cite{edsurge2014}. Moreover, \textit{flipped} classrooms, where students watch pre-recorded lectures outside of class, and do more active learning activities within class--are becoming increasingly popular in schools.  

Despite the increasingly important and broad role videos are taking in education, video is not an inherently easy medium to navigate dense information. It is difficult for viewers to browse and skim through the underlying content. Viewers have to scrub back-and-forth on a linear timeline slider in order to gain an overview of the topics, or locate specific information of interest. Many platforms, such as Khan Academy and YouTube provide synchronized transcripts, which allow users to click on a word in the transcript and play the video at that location. While transcripts do expose important content from the video, without structured organization and correspondence with visual content, long blocks of text are time-consuming to read and difficult to skim. 

Educational videos are different from traditional video contents such as movies. They are usually shorter, more static (less change in scenes), and denser in information, especially textual information. Learning also involves a variety of different interactions with the video. These interactions include: (1) skimming to get a quick overview of the video, (2) searching to find a specific information in the lecture, and (3) re-watching to review specific portions of the video. However, there has been few attempts to adapt video interfaces to support these new types of needs. Exploring the design space of interfaces fitted to the contents and navigation patterns specific to education can facilitate and improve learning experience.

This paper presents ScribeVideo, a readable, tutorial-style re-presentation of videos to facilitate browsing and navigation of blackboard-style lectures. \todo{Advantage of static, tutorial-style.} The key insight of our approach derives from the following characteristics of blackboard-style lectures. 
\begin{itemize}
\item \textbf{Lectures present information sequentially.} Concepts are introduced one by one, or an idea is developed progressively (e.g. derivation of an equation, or explanation of a chemical procedure). Information presented in such structured ways lends itself naturally to a linear format.
\item \textbf{Most of the visual content convey static information.} The dynamic nature of videos means that visuals are also presented progressively (i.e. they are drawn in live-time). However, the resultant visual that actually conveys information is usually static (e.g. graphs, equations, figures as opposed to animation). This, along with the previous point, means there are \textit{chunks} of visual content, where each chunk is more naturally presented in a static manner, but consecutive chunks should be presented sequentially.
\item \textbf{There is a close correspondence between visual and audio content.} In a video, this correspondence is naturally communicated through the medium of time. Visuals are explained at the same time as they are being drawn. In a static format, this correspondence is displayed in space through layout.
\end{itemize}

\todo{Explain ScribeNote. ScribeNote interleaves figures (visual chunk) with corresponding texts in a sequential layout.}

\todo{Brief explanation of comparison in user study, and result.}

\section{Related Work}

\textbf{Video Visualization} Creating new visual representations from an input video to reveal important features and events in the video is a long-standing research problem. There is a large body of work that aims to automatically summarize videos to facilitate navigation and browsing. Recent survey papers \cite{truong2007video,borgo2011survey} comprehensively review these techniques, which can be broadly divided into two classes according to their output: \textit{video skims} and \textit{still-image abstracts}. Video skims \cite{he1999auto,ekin2003automatic,ngo2005video,lu2013story} summarize a longer video with a shorter video, usually consisting of segments extracted from the original video. These skims have the advantage of retaining audio and motion elements that is especially useful for expressing dynamic scenes, but they are less suitable for conveying dense, static information and still rely on timeline-based navigation. Still-image based methods \cite{uchihashi1999video,barnes2010video,hwang2006cinema,boreczky2000interactive}
primarily focus on conveying the visual content of a
video in static form through a collection of salient images extracted from the video. Most relevant to our work is \cite{choudary2007summarization}, which summarizes blackboard-style lectures by creating a panoramic frame of the board.

\textbf{Video Navigation Interface}
\cite{kim2014data} uses interaction data collected from MOOC platforms to introduce a set of techniques that augment existing video interface widgets. \cite{pavel2014video} provides a tool to create \texit{video digests}, structured summaries of informational videos organized into chapters and sections. 
Most closely related to our work is the NoteVideo interface by \cite{monserrat2013notevideo}, which presents a summarized image of the video composed of clickable visual links to specific places in  lecture.

\textbf{Improving Transcripts / Subtitles}
\cite{hu2015speaker} optimizing subtitle placement by speaker recognition. \cite{kurlander1996comic}, \cite{chun2006automated} subtitle placement for single frame in comic style cinema.




\section{Method}
\subsection{Time-stamped Transcript}
Several on-line video lectures (e.g. Khan Academy) come with transcripts. In cases where transcripts were not provided, we used an on-line audio transcription service to acquire a verbatim text transcript. Then, we used a tool from \cite{rubin2013content} to compute an alignment between the video's audio file and the transcript. The final output is a time-stamped transcript, where each word is annotated with a start and end time.

\subsection{Stroke Extraction}
A \textit{stroke} is defined as the set of foreground pixels that is drawn during one continuous drawing activity. The method used to extract strokes from video frames are similar to that used by \cite{monserrat2013notevideo} to extract visual objects in their NoteVideo interface. Figure~\ref{Fig:stroke_examples} shows examples of extracted strokes from different videos. A typical stroke comprises several characters to several words, or it can also be a part of other drawings such as a graph (Figure~\ref{Fig:stroke_examples}c).  

\begin{figure}[h]
       \centering
        \includegraphics[width=0.45\textwidth, clip=true]{images/example_strokes}
        \caption{Examples of strokes extracted from different videos.}
        \label{Fig:stroke_examples}
\end{figure}

\subsection{Hierarchical Grouping of Strokes}
We group strokes into hierarchical units: lines and sentence-strokes.  
A line consists of a set of strokes that \textit{belong together} semantically. For example, a line could be a single row of equations, or a graph including its labels. Figure~\ref{Fig:line_examples} shows examples of lines. The problem of grouping strokes into lines is analogous to the problem of line breaking, also known as word wrapping \cite{knuth1981breaking}. An important difference is that in the traditional word wrapping problem, only a contiguous set of words can be put in the same line. In our case, strokes in a single line can be interspersed by strokes in a different line. This happens, for example, when the instructor goes back and forth between two lines of equations, or between a graph and an equation (Figures~\ref{Fig:line_order}). 

Scoring function description

Pseudo code figure
\begin{algorithm}[h]
\DontPrintSemicolon
\SetAlgoLined
\SetCommentSty{\small\ttfamily} 
  \SetKwInOut{Input}{Input}\SetKwInOut{Output}{Output}
  \Input{list of strokes ${S}$}
  \Output{list of optimal lines, ${L_{\left\vert{S}\right\vert}}$}
  $L_{-1} =$ \string{\string} \tcp{$L_i=$ optimal set of lines upto $i$-th stroke}\;
    \For{each stroke $s_i \in \mathbf{S}$}{
    \textit{minscore} = $+\infty$\;
    \For{$j \leftarrow -1$ \textbf{to} $i-1$}{
      \For{$n \leftarrow 0$ \textbf{to} $\left\vert{L_j}\right\vert+1$}{
        \tcp{score to merge $s_i$ to $n$-th line of $L_j$}\;
        \tcp{If $n = \left\vert{L_j}\right\vert+1$, $s_i$ on a new line by itself.}\;
        \textit{score} $\leftarrow$ line\_score($L_{j}$, $n$)\;
        \If {score \textless minscore}{
          ${optj} = j$\;
          ${optn} = n$\;
        }
      }
   }
   $L_i=$ merge $s_i$ to $optn$-th line of $L_{optj}$\;
}

\end{algorithm}

\begin{figure*}[h]
        \centering
        \includegraphics[width=0.9\textwidth]{images/example_lines}
        \caption{Examples of lines (i.e. set of strokes that belong together semantically) output from our line-breaking algorithm. Our algorithm successfully identifies meaningful groups even from complex layouts with a mix of equations, figures and graphs.}
        \label{Fig:line_examples}
\end{figure*}

Sentence is a meaningful unit. So, we divide the strokes in a line to sentence-strokes.

In summary, we have the following hierarchical grouping of strokes: strokes, sentence-strokes, and lines.



\subsection{Layout}



\input{results}
\section{User Study}
\subsection{Comprehension Task}
\cite{monserrat2014live}: 10-question pre-test and 10-question post-test. The two sets of questions test the same type of knowledge but were asked in a different ways (e.g. one question would be asked in identification form with a why question after, and the other question would be in essay form). Measured test scores and learning time.
\subsection{Overview Task}
\subsection{Search Task}
\input{conclusion}
\section*{Acknowledgements}


\bibliographystyle{acmsiggraph}
\nocite{*}
\bibliography{paper}
\end{document}
