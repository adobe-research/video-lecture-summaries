\section{Introduction}
With the introduction of online education platforms such as Khan Academy, Coursera, Udacity, edX, and online courses offered by universities, video watching has become an integral part of the students' learning experience. For example, in 2014, the number of universities offering MOOCs doubled to 400 universities offering over 2,400 courses to 16-18 million enrolled students \cite{edsurge2014}. Moreover, \textit{flipped} classrooms, where students watch pre-recorded lectures outside of class, and do more active learning activities within class--are becoming increasingly popular in schools.  

Despite the increasingly important and broad role videos are taking in education, video is not an inherently easy medium to navigate dense information. It is difficult for viewers to browse and skim through the underlying content. Viewers have to scrub back-and-forth on a linear timeline slider in order to gain an overview of the topics, or locate specific information of interest. Many platforms, such as Khan Academy and YouTube provide synchronized transcripts, which allow users to click on a word in the transcript and play the video at that location. While transcripts do expose important content from the video, without structured organization and correspondence with visual content, long blocks of text are time-consuming to read and difficult to skim. 

Educational videos are different from traditional video contents such as movies. They are usually shorter, more static (less change in scenes), and denser in information, especially textual information. Learning also involves a variety of different interactions with the video. These interactions include: (1) skimming to get a quick overview of the video, (2) searching to find a specific information in the lecture, and (3) re-watching to review specific portions of the video. However, there has been few attempts to adapt video interfaces to support these new types of needs. Exploring the design space of interfaces fitted to the contents and navigation patterns specific to education can facilitate and improve learning experience.

This paper presents ScribeVideo, a static, linear re-presentation of videos to facilitate navigation of blackboard-style lectures. 

