\section{Introduction}
With the introduction of online education platforms such as Khan Academy, Coursera, Udacity, edX, and online courses offered by universities, video watching has become an integral part of the students' learning experience. For example, in 2014, the number of universities offering MOOCs doubled to 400 universities offering over 2,400 courses to 16-18 million enrolled students \cite{edsurge2014}. Moreover, \textit{flipped} classrooms, where students watch pre-recorded lectures outside of class, and do more active learning activities within class--are becoming increasingly popular in schools.  

Despite the increasingly important and broad role videos are taking in education, there has been few development in video navigation techniques to support the new needs. Educational contents are different from traditional video contents such as movies. They are usually shorter, more static (less change in scenes), and denser in information, especially textual information. Learning also involves a variety of different interactions with the video. These interactions include: (1) skimming to get a quick overview of the video, (2) searching to find a specific information in the lecture, and (3) re-watching to review specific portions of the video. 

This paper ScribeVideo, a static, linear format to present blackboard-style lecture videos, which facilitates navigation of their content.

